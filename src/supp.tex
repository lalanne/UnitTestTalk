\begin{frame}[fragile]{How does it look like in C like C++}
    \begin{columns}[T]
        \begin{column}[T]{5cm}
            \inputminted[mathescape,
                       linenos,
                       numbersep=2pt,
                       frame=lines,
                       bgcolor=White,
                       fontsize=\tiny,
                       linenos,
                       framesep=1mm]{c++}
                       {/home/elf/PersonalProjects/MetaTalk/src/code/dd_c1.cpp} 
        \end{column}
        \begin{column}[T]{5cm}
            \inputminted[mathescape,
                   linenos,
                   numbersep=2pt,
                   frame=lines,
                   bgcolor=White,
                   fontsize=\tiny,
                   linenos,
                   framesep=1mm]{c++}
                   {/home/elf/PersonalProjects/MetaTalk/src/code/dd_c12.cpp}
        \end{column}
    \end{columns}
\end{frame}


\begin{frame}{Virtual Tables} 
    \begin{itemize}
        \item This virtual table in turn contains the base addresses of one or more virtual functions of the class. At the time when a virtual function is 
            called on an object, the vptr of that object provides the base address of the virtual table for that class in memory. This table is used to resolve 
            the function call as it contains the addresses of all the virtual functions of that class. This is how dynamic binding is resolved during a virtual 
            function call.
    \end{itemize}
\end{frame}

